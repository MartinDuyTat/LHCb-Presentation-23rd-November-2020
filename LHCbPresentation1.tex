%%%%%%%%%%%%%%%%%%%%%%%%%%%%%%%%%%%%%%%%%%%%%%%%%%%%%%%%%%%%%%%%%%%%%%
% Overleaf (WriteLaTeX) Example: Molecular Chemistry Presentation
%
% Source: http://www.overleaf.com
%
% In these slides we show how Overleaf can be used with standard 
% chemistry packages to easily create professional presentations.
% 
% Feel free to distribute this example, but please keep the referral
% to overleaf.com
% 
%%%%%%%%%%%%%%%%%%%%%%%%%%%%%%%%%%%%%%%%%%%%%%%%%%%%%%%%%%%%%%%%%%%%%%

\documentclass{beamer}

\mode<presentation>
{
  \usetheme{Madrid}       % or try default, Darmstadt, Warsaw, ...
  \usecolortheme{default} % or try albatross, beaver, crane, ...
  \usefonttheme{serif}    % or try default, structurebold, ...
  \setbeamertemplate{navigation symbols}{}
  \setbeamertemplate{caption}[numbered]
} 

\usepackage[english]{babel}
\usepackage[utf8x]{inputenc}
\usepackage{chemfig}
\usepackage[version=3]{mhchem}

% Here's where the presentation starts, with the info for the title slide
\title[Molecules in \LaTeX{}]{A short presentation on molecules in \LaTeX{}}
\author{J. Hammersley}
\institute{www.overleaf.com}
\date{\today}

\begin{document}

\begin{frame}
  \titlepage
\end{frame}

% These three lines create an automatically generated table of contents.
\begin{frame}{Outline}
  \tableofcontents
\end{frame}

\section{Introduction}

\begin{frame}{Introduction}

\begin{itemize}
  \item In these slides we show how Overleaf can be used with standard chemistry packages to easily create professional presentations.
  \item If you're new to \LaTeX{}, check out this free introductory course by Overleaf founder Dr John Lees-Miller: \url{www.overleaf.com/blog/7}
  \item You can also find more quick tips and tricks on the help pages at \url{www.overleaf.com/help}
\end{itemize}

\begin{center}\small\setchemfig{atom sep=1.5em}
\schemestart
  \chemfig{*6(=-*6(-\chembelow{N}{H}-NH_2)=-=-)}
  \+
  \chemfig{(=[:-150]O)(-[:-30]R_2)-[2]-[:150]R_1}
  \arrow(.mid east--.mid west){->[\chemfig{H^+}]}
  \chemfig{*6(-=*5(-\chembelow{N}{H}-(-R_2)=(-R_1)-)-=-=)}
\schemestop
\end{center}

\end{frame}

\subsection{The chemistry packages}
\begin{frame}{The chemistry packages}

We focus on two \LaTeX{} chemistry packages:
\begin{block}{The \texttt{chemfig} package}
This package provides the command which draws molecules. Created by Christian Tellechea, a detailed user guide can be found here:\\[0.4cm]
\small{\url{www.tex.ac.uk/ctan/macros/generic/chemfig/chemfig_doc_en.pdf}}
\end{block}
\begin{block}{The \texttt{mhchem} package}
The \texttt{mhchem} package provides simple commands for typesetting chemical molecular formulae and equations. Created by Martin Hensel, a detailed user guide can be found here:\\[0.4cm]
\small{\url{http://mirror.ox.ac.uk/sites/ctan.org/macros/latex/contrib/mhchem/mhchem.pdf}}
\end{block}

\end{frame}

\section{Using chemistry packages with \LaTeX{}}

\subsection{Chemical equations with \texttt{mhchem}}

\begin{frame}[fragile]
\frametitle{Chemical equations with \texttt{mhchem}}

\begin{itemize}
\item The \texttt{mhchem} package lets you write chemical equations in \LaTeX{} with the minimum of effort. 
\item The example below shows how the standard representation of a reaction (on the left) is created from the simple code on the right:
\end{itemize}

\begin{center}
\ce{CO2 + C -> 2CO} is created with \verb|\ce{CO2 + C -> 2CO}|
\end{center}

\begin{itemize}
\item More complicated reactions are still easy to write:
\end{itemize}

\begin{center}
\ce{SO4^2- + Ba^2+ -> BaSO4 v}\\[0.1cm]
is created with\\[0.1cm]
\verb|\ce{SO4^2- + Ba^2+ -> BaSO4 v}|
\end{center}

\end{frame}

\section{Where to go next\dots{}}

\begin{frame}{Where to go next\dots{}}

\begin{itemize}
\item This short example was designed to introduce you to using Overleaf for scientific presentations.
\item This is made possible by the many great packages that have been developed for \LaTeX{}, including the two we focused on here (plus the \texttt{Beamer} package used for the overall presentation style). 
\item For more help on using \LaTeX{}, see the links on the Overleaf help page: \url{www.overleaf.com/help} or check out our free introductory course: \url{www.overleaf.com/blog/7}.
\end{itemize}

\begin{center}
Follow @overleaf on Twitter for all the latest news and updates.\\[0.3cm]
Happy \LaTeX ing!
\end{center}

\end{frame}

\end{document}
